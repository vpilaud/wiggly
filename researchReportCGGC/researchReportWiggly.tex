\documentclass{amsart}

\usepackage{amsmath, amsfonts, amssymb, amsthm, thmtools, graphics, graphicx, xcolor, url, hyperref, hypcap, a4wide, xargs}
\hypersetup{colorlinks=true, citecolor=darkblue, linkcolor=darkblue}
\usepackage{tikz}
\usepackage[noabbrev,capitalise]{cleveref}
\usepackage{subcaption}
\usepackage{tikz, tikz-3dplot}
\usepackage{arcs}
\usetikzlibrary{math, calc, arrows, arrows.meta} % LIBRARY NEEDED FOR PICTURE
\usetikzlibrary{tikzmark, bending}

\newcommand{\arcThroughThreePoints}[4][]{
	\coordinate (middle1) at ($(#2)!.5!(#3)$);
	\coordinate (middle2) at ($(#3)!.5!(#4)$);
	\coordinate (aux1) at ($(middle1)!1!90:(#3)$);
	\coordinate (aux2) at ($(middle2)!1!90:(#4)$);
	\coordinate (center) at ($(intersection of middle1--aux1 and middle2--aux2)$);
	\draw[#1] 
	let \p1=($(#2)-(center)$),
	\p2=($(#4)-(center)$),
	\n0={veclen(\p1)},       % Radius
	\n1={atan2(\y1,\x1)}, % angles
	\n2={atan2(\y2,\x2)},
	\n3={\n2>\n1?\n2:\n2+360}
	in (#2) arc(\n1:\n3:\n0);
}

\graphicspath{{figures/}}
\makeatletter
\def\input@path{{figures/}}
\makeatother

%%%%%%%%%%%%%%%%%%%%%%%%%%%%%%%%%%%%%%

\title{Wigglyhedra for planar point sets}

\author[D. Alcantara]{David Alcantara}
\address[David Alcantara]{Universidad de Cantabria, Santander, Spain}
\email{david.alcantara@unican.es}
%\urladdr{\url{}}

\author[A. Bapat]{Asilata Bapat}
\address[Asilata Bapat]{The Australian National University, Canberra, Australia}
\email{asilata.bapat@anu.edu.au}
\urladdr{\url{https://asilata.github.io}}

\author[M. Bin]{Marguerite Bin}
\address[Marguerite Bin]{Université de Lorraine, CNRS, INRIA, LORIA, Nancy, France}
\email{marguerite.bin@inria.fr}
%\urladdr{\url{}}

\author[N. Crepeau]{Natasha Crepeau}
\address[Natasha Crepeau]{University of Washington, Seattle, USA}
\email{ncrepeau@uw.edu}
\urladdr{\url{https://sites.google.com/view/natasha-crepeau-math/home}}

\author[H. Ping Luk]{Hoi Ping Luk}
\address[Hoi Ping Luk]{Z\'apado\v{c}esk\'a univerzita v Plzni, Pilsen, Czech Republic}
\email{hoi@connect.ust.hk}
%\urladdr{\url{}}

\author[V. Pilaud]{Vincent Pilaud}
\address[Vincent Pilaud]{Universitat de Barcelona \& Centre de Recerca Matemàtica, Barcelona, Spain}
\email{vincent.pilaud@ub.edu}
\urladdr{\url{https://www.ub.edu/comb/vincentpilaud/}}

\author[F. Santos]{Francisco Santos}
\address[Francisco Santos]{Universidad de Cantabria, Santander, Spain}
\email{francisco.santos@unican.es}
\urladdr{\url{https://personales.unican.es/santosf/}}

%%%%%%%%%%%%%%%%%%%%%%%%%%%%%%%%%%%%%%

% theorems
\newtheorem{theorem}{Theorem}
\newtheorem{proposition}[theorem]{Proposition}
\newtheorem{lemma}[theorem]{Lemma}
\newtheorem{corollary}[theorem]{Corollary}
\newtheorem{conjecture}[theorem]{Conjecture}
\crefname{conjecture}{Conjecture}{Conjectures}

\theoremstyle{definition}
\newtheorem{definition}[theorem]{Definition}
\newtheorem{example}[theorem]{Example}
\newtheorem{remark}[theorem]{Remark}
\newtheorem{question}[theorem]{Question}
\newtheorem{openproblem}[theorem]{Open problem}


% newcommands
% math special letters
\newcommand{\C}{\mathbb{C}} % complexes
\newcommand{\R}{\mathbb{R}} % reals
\newcommand{\N}{\mathbb{N}} % naturals
\newcommand{\Z}{\mathbb{Z}} % integers
\renewcommand{\b}[1]{\boldsymbol{#1}} % bold
\renewcommand{\c}[1]{\mathcal{#1}} % bold

% math commands
\newcommand{\set}[2]{\left\{ #1 \;\middle|\; #2 \right\}} % set notation
\newcommand{\pth}[1]{\left( #1 \right)} % set notation
\newcommand{\ssm}{\smallsetminus} % small set minus
\newcommand{\dotprod}[2]{\langle #1 | #2 \rangle} % dot product
\newcommand{\symdif}{\triangle} % symmetric difference
\newcommand{\one}{{1\!\!1}} % the all one vector
\newcommand{\zero}{{\bf 0}} % the all zero vector
\newcommand{\norm}[1]{{\left\lVert{#1}\right\rVert}} % vector norm
\newcommand{\abs}[1]{\left\lvert{#1}\right\rvert} % absolute value
\newcommand{\cardinal}[1]{\left\lvert{#1}\right\rvert} % set cardinal
\newcommand{\sign}{{\textnormal{sign}\,}} % sign operator
\newcommand{\eqdef}{\mbox{\,\raisebox{0.2ex}{\scriptsize\ensuremath{\mathrm:}}\ensuremath{=}\,}} % :=
\newcommand{\defeq}{\mbox{~\ensuremath{=}\raisebox{0.2ex}{\scriptsize\ensuremath{\mathrm:}} }} % =:

% specific wiggly
\newcommandx{\arc}[1][1=\alpha]{\b{#1}} % arc
\newcommand{\wigglyArcs}{\mathrm{WA}} % wiggly complex
\newcommand{\extendedWigglyComplex}{\overline{\mathrm{WC}}} % wiggly complex
\newcommand{\reducedWigglyComplex}{\mathrm{WC}} % wiggly complex
\newcommand{\wigglyFlipGraph}{\mathrm{WFG}} % wiggly flip graph
\newcommand{\wigglyIncreasingFlipGraph}{\mathrm{WIFG}} % wiggly increasing flip graph
\newcommand{\wigglyLattice}{\mathrm{WL}} % wiggly lattice
\newcommand{\wigglyFan}{\mathrm{WF}} % wiggly fan
\newcommand{\wigglyhedron}{\polytope{W}} % wigglyhedron
\newcommand{\Asso}{\polytope{A}\mathsf{sso}} % associahedron


% others
\newcommand{\fix}[1]{{\bf FIXME: }#1} % emphasis of a problem to FIX
\newcommand{\ie}{\textit{i.e.}~} % id est
\newcommand{\eg}{\textit{e.g.}~} % exempli gratia
\newcommand{\Eg}{\textit{E.g.}~} % exempli gratia
\newcommand{\aka}{\textit{aka.}~} % also known as
\newcommand{\viceversa}{\textit{vice versa}} % vice versa
\definecolor{darkblue}{rgb}{0,0,0.7} % darkblue color
\newcommand{\defn}[1]{\textsl{\color{darkblue} #1}} % emphasis of a definition

%%%%%%%%%%%%%%%%%%%%%%%%%%%%%%%%%%%%%%

\begin{document}

\vspace*{-2cm}
\begin{center}
\small
Research Report for the Intensive Research Program \\ Combinatorial Geometries and Geometric Combinatorics 2025
\end{center}

\maketitle

We consider an arbitrary point set $P$ in the plane
(neither necessarily aligned, nor necessarily in general position).

\begin{definition}
    \label{def:wiggly-segment}
    A \defn{wiggly curve} on $P$ is a curve with endpoints in $P$, whose interior contains no point
    of $P$, considered up to homotopy.
    The \defn{length} of a wiggly curve is the infimum of the length of its homotopy representatives.
    A \defn{wiggly segment} on $P$ is a wiggly curve joining $p, j\in P$ of length $\mid p - q\mid$.
    In other words, it just decides a side for each point in the interior of the straight segment
    joining $p$ to $q$.
    
    Given a generic linear functional $c: \R^2 \to \R$, \ie, such that it evaluates differently
    for every two points of $P$, we denote by $(p, q, A, B)_c$ the wiggly segment between
    $p$ and $q$ where $c\cdot q > c\cdot p$, and the set $A$ (resp. $B$) contain the points of $P$
    within the straight segment $\overline{pq}$ that the wiggly segment leaves \emph{above}
    (resp. \emph{below}) with respect to $c$.
    When the functional $c$ is irrelevant, we will omit it.
\end{definition}

We denote by $\wigglyArcs_P$ the set of all wiggly segments on $P$.

\begin{definition}
    \label{def:wiggly-pseudodissection}
    A \defn{wiggly pseudodissection} of $P$ is a collection $X$ of wiggly segments which is
    \begin{itemize}
        \item \defn{non-crossing}: no pair of wiggly segments of $X$ cross in their interior,
        \item \defn{pointed}: for any $p\in P$, the wiggly segments of $X$ with an endpoint at
        $p$ generate a pointed cone (meaning that they are included in an open halfplane).
    \end{itemize}
\end{definition}

The \defn{wiggly complex} $\reducedWigglyComplex_P$ is the simplicial complex of wiggly pseudodissections.
Note that the boundary wiggly segments are irrelevant, which allows us to consider a reduced wiggly
complex $\reducedWigglyComplex_P$ induced by internal wiggly segments.

A \defn{wiggly pseudotriangulation} of $P$ is a facet of $\reducedWigglyComplex_P$.
The \defn{wiggly flip graph} $\wigglyFlipGraph_P$ is the adjacency graph of the facets of $\reducedWigglyComplex_P$.

Note that Definition~\ref{def:wiggly-pseudodissection} recovers two specific situations, namely
\begin{itemize}
    \item the wiggly complex of~\cite{BapatPilaud25} in the case of aligned points, and
    \item the pseudodissection complex of $P$ and pseudotriangulation flip graph of $P$ in the case
    of points $P$ in general position~\cite{PocchiolaVegter96, RoteSantosStreinu08}.
\end{itemize}

The general case of Definition~\ref{def:wiggly-pseudodissection} was very recently studied by
A. Bapat, A. Deopurkar, A. Licata in~\cite{BapatDeopurkarLicata25}, in connection to Bridgeland
stability conditions on triangulated categories.
They prove in particular that the wiggly complex $\reducedWigglyComplex_P$ is a sphere
(hence that the wiggly flip graph is well-defined).

\vspace*{1em}%

In this project, we propose a realization of the normal fan of the
wiggly complex for general planar point sets that we have checked to be
well defined and polytopal for several point sets.

We also make progress towards proving its polytopality (and well-definition as a fan),
by proving that the hypotheses of Theorem~\ref{thm:simplicial-complex-is-regular-triangulation}
hold for this realization in some cases.

\begin{theorem}[Santos]
    \label{thm:simplicial-complex-is-regular-triangulation}
    Let $\Delta$ be a simplicial complex that is pure of dimension $d - 1$ with
    a $d$-regular dual graph, with vertex set $V$.
    Let $R: V \to \R^d$ be a vector configuration labelled by $V$,
    and $f: V \to \R$, a \emph{height function} or \emph{rhs function}.
    
    Then $\Delta$ is the regular triangulation of $R(V)$ w.r.t. $f$
    if and only if for every minimum non-face $F$ of $\Delta$ there is a linear
    dependence $\lambda_F\in \R^V$ of $R$ such that:
    \begin{itemize}
        \item $\lambda_F^{-} = F$,
        \item $\lambda_F^{+}\in \Delta$, that is, it is a face of $\Delta$,
        \item $\lambda_F \cdot f > 0$,
    \end{itemize}
    where $\lambda_F^{\pm} := \{v: \sign{\ell_v} = \pm\}$.
\end{theorem}

\section{Realization of the wiggly complex}
We propose a realization of the wiggly complex defined as an $\epsilon$-deformation of the
rigidity-based realization for pointed pseudotriangulations in~\cite{RoteSantosStreinu03},
that incorporates the combinatorial encoding of wiggly compatibilities for points in a straight
line from~\cite{BapatPilaud25} as an $\epsilon$ factor.

\begin{definition}
    \label{d:rigidity-vector}
    Let $\epsilon > 0$.
    The \defn{rigidity vector} of a wiggly segment $\alpha = (p, q, A, B)$ is defined as
    $R_\epsilon: \wigglyArcs_P \to (\R^2)^P$, with $R_\epsilon(\alpha) = r(\alpha) + \epsilon w(\alpha)$, where
    \[
        r(\alpha)_v =
        \begin{cases}
            \hphantom{+}\vec{\alpha}_1&\text{if } v = p\\
            -\vec{\alpha}_1&\text{if } v = q\\
            \zero&\text{otherwise},
        \end{cases}
        %
        \quad\quad
        %
        w(\alpha)_v =
        \begin{cases}
            -(\norm{v - p} + \norm{q - v})\, \vec{\alpha}_2&\text{if } v\in A\\
            \hphantom{+}(\norm{v - p} + \norm{q - v})\, \vec{\alpha}_2&\text{if } v\in B\\
            \left(\sum_{v\in A} \norm{q - v}\; - \;\sum_{v\in B} \norm{q - v}\right)\, \vec{\alpha}_2&\text{if } v = p\\
            \left(\sum_{v\in A} \norm{v - p}\; - \;\sum_{v\in B} \norm{v - p}\right)\, \vec{\alpha}_2&\text{if } v = q\\
            \zero&\text{otherwise},
        \end{cases}
    \]
    where $\vec{\alpha}_1 := q - p$ and $\vec{\alpha}_2 := (p_y - q_y,\, q_x - p_x)$,
    the orthogonal rotation of $\vec{\alpha}_1$.
\end{definition}

While the rigidity vector is defined as a set of vectors in $\R^2$ indexed by $P$,
it can be naturally embedded as a vector of $\R^{2\cardinal{P}}$.

The term $r(\alpha)$ is precisely the rigidity vector in the sense of classical rigidity for
generic point sets in $\R^2$.
It describes the longitudinal stress induced by $\alpha$ on its endpoints.
The term $w(\alpha)$ describes the orthogonal stress induced by the wiggly segment $\alpha$ on its endpoints
and the points it wefts around.

\begin{definition}
    The \defn{rigidity wiggly fan} of a point set $P$ is defined as the fan in $\R^{2\cardinal{P}}$
    with rays given by the rigidity vectors of all wiggly segments in $\wigglyArcs_P$,
    and cones given by the faces of $\reducedWigglyComplex_P$.
\end{definition}

When $\alpha$ is a non-wiggly segment, we have that $R_\epsilon(\alpha) = r(\alpha)$.
Hence, when $P$ is a generic point set, the rigidity wiggly fan coincides with the
rigidity fan from~\cite{RoteSantosStreinu03}, and is thus polytopal.

In our aim to prove that the rigidity wiggly fan is indeed a fan, and a polytopal realization
of $\reducedWigglyComplex_P$, we have proven the following result:

\begin{lemma}
    \label{lemma:wiggly-rigidity-fan-for-aligned-points}
    Let $\alpha$ and $\beta$ be two incompatible wiggly segments in an aligned point set $P$.
    
    Then, there is a unique linear dependence $\lambda$ on $R_\epsilon(P)$, such that
    \begin{enumerate}
        \item $\lambda^- = \{\alpha, \beta\}$,
        \item $\lambda^+ \in \Delta$
    \end{enumerate}
\end{lemma}

\begin{definition}
    Let $c$ be a generic linear functional on $P$.
    The $c-$\defn{upper convex hull} and $c-$\defn{lower convex hull} of a collection $X$
    of wiggly segments are defined as {\bf \color{red} [definition in terms of truncations]}.
\end{definition}

\begin{proof}[Proof of Lemma~\ref{lemma:wiggly-rigidity-fan-for-aligned-points}]
    {\bf \color{red} [linear dependence with the convex hull using truncations on both cases: crossing, non-pointed]}
\end{proof}

\section{Future questions}
\begin{question}
    Is there a dual interpretation of wiggly pseudotriangulations as some sort of pseudoline arrangements?
\end{question}

\begin{question}
    Is there a multi wiggly complex?
\end{question}

\begin{question}
    Can the wiggly complex be extended to arbitrary finite Coxeter groups?
\end{question}

\begin{question}
    Is the wiggly flip graph $\wigglyFlipGraph_P$ Hamiltonian for any point set $P$ in the plane?
\end{question}


\section{The aligned case}

The aligned case has previously been studied in \cite{BapatPilaud25}). Here we give another construction of a fan, by choosing rigidity vectors (i.e a function $R:\wigglyArcs_P\to \R^d$) that satisfies the hypotheses of Theorem~\ref{thm:simplicial-complex-is-regular-triangulation}. We believe the fans to be different.
%
We begin by defining the function $R$ on a set of ``elementary'' wiggly segments
\[
\widetilde {\wigglyArcs_P}:=\set{(\square,i)}{\square \in \{l,r,u,d\}, i\in \{2,\ldots,n-1\}}.
\]
For these elementary segments, it is easy to choose $R$ to ensure that the hypotheses of Theorem~\ref{thm:simplicial-complex-is-regular-triangulation} are satisfied.
%
Each wiggly segment $\alpha$ is then associated with a formal sum of elementary wiggly segments, denoted $D(\alpha)$. This allows us to extend the function $R$ to all wiggly segments. 

\begin{definition}
	Let $\alpha$ be a wiggly arc. A \textit{truncation} of $\alpha$ at $i$ is defined in the following way:
	\begin{enumerate}
		\item Fix a frame / window starting at the $i-1$-st node and ending at the $i+1$-st node;
		\item Draw the segment of the chosen wiggly arc in the interior of the frame and draw the endpoints of the wiggly arc to the $i-1$-st node and the $i+1$-st node. 
	\end{enumerate}
	We denote the truncation of $\alpha$ around $i$ as $Tr_i(\alpha)$.
\end{definition}



\begin{example}
	Figure \ref{Fig-trunc-eg1} shows two crossing wiggly arcs $\alpha$ and $\alpha'$. Another two arcs $\beta$ and $\beta'$ are added to create a dependent set.
	\begin{figure}[h!]
		\centering
		\begin{subfigure}[t]{\linewidth}
			\centering
			\begin{tikzpicture}
				\tikzmath{
					\x=1;
					\rr=0.08*\x;
					\YS=2.5;
				}
				
				\arcThroughThreePoints[]{-0.5,0}{0.5,-0.5}{1,0}; 
				\arcThroughThreePoints[]{-0.5,0}{-1,0.5}{-2,0}; 
				\arcThroughThreePoints[]{2,0}{1,0.5}{-1,0};
				
				\arcThroughThreePoints[dashed]{2,0}{0,1.5}{-2,0};
				\arcThroughThreePoints[dashed]{-1,0}{0,-0.75}{1,0};
				
				\fill
				(0,0) circle (\rr)
				;
				\foreach \aa in {-1,1} {
					\tikzset{xscale=\aa}
					\fill
					(\x,0) circle (\rr)
					(2*\x,0) circle (\rr)
					;
				}
				
				\node at (-1*\x, 0.75*\x) {\small $\alpha$};
				\node at (1*\x, 0.75*\x) {\small $\alpha'$};
				
				\node at (0, 1.75*\x) {\small $\beta$};
				\node at (0, -\x) {\small $\beta'$};
				
				\node at (-2*\x,-0.25*\x) {\scriptsize $1$}; 
				\node at (-1*\x,-0.25*\x) {\scriptsize $2$}; 
				\node at (0*\x,-0.25*\x) {\scriptsize $3$}; 
				\node at (1*\x,-0.25*\x) {\scriptsize $4$}; 
				\node at (2*\x,-0.25*\x) {\scriptsize $5$}; 
			\end{tikzpicture}
			\caption{}
			\label{Subfig-trunc-eg1}
		\end{subfigure}
		\begin{subfigure}[t]{0.24\linewidth}
			\centering
			\begin{tikzpicture}[scale=0.65]
				\tikzmath{
					\x=1;
					\rr=0.08*\x;
					\YS=2.5;
				}
				
				\begin{scope}[yshift=-\YS cm]
					
					\arcThroughThreePoints[]{1,0}{0,0.5}{-1,0};
					
					\fill
					(-1,0) circle (\rr)
					(0,0) circle (\rr)
					(1,0) circle (\rr)
					;
					
					\node at (-2.25*\x,0) {\scriptsize $Tr_2(\alpha):$};
					
					\node at (-1*\x,-0.3*\x) {\scriptsize $1$}; 
					\node at (0*\x,-0.3*\x) {\scriptsize $2$}; 
					\node at (1*\x,-0.3*\x) {\scriptsize $3$}; 
					
				\end{scope}
				
				\begin{scope}[yshift=-1.5*\YS cm]
					
					\arcThroughThreePoints[]{-1,0}{0,-0.5}{1,0};
					
					\fill
					(-1,0) circle (\rr)
					(0,0) circle (\rr)
					(1,0) circle (\rr)
					;
					
					\node at (-2.25*\x,0) {\scriptsize $Tr_3(\alpha):$};
					
					\node at (-1*\x,-0.3*\x) {\scriptsize $2$}; 
					\node at (0*\x,-0.3*\x) {\scriptsize $3$}; 
					\node at (1*\x,-0.3*\x) {\scriptsize $4$}; 
					
				\end{scope}
				
				\begin{scope}[yshift=-2*\YS cm]
					
					\draw[]
					(-1*\x,0) -- (0,0)
					;
					
					\fill
					(-1,0) circle (\rr)
					(0,0) circle (\rr)
					(1,0) circle (\rr)
					;
					
					\node at (-2.25*\x,0) {\scriptsize $Tr_4(\alpha):$};
					
					\node at (-1*\x,-0.3*\x) {\scriptsize $3$}; 
					\node at (0*\x,-0.3*\x) {\scriptsize $4$}; 
					\node at (1*\x,-0.3*\x) {\scriptsize $5$}; 
					
				\end{scope}
				
			\end{tikzpicture}
			\caption{$Tr_i(\alpha)$}
			\label{Subfig-trunc-eg1-al}
		\end{subfigure}
		\begin{subfigure}[t]{0.24\linewidth}
			\centering
			\begin{tikzpicture}[scale=0.65]
				\tikzmath{
					\x=1;
					\rr=0.08*\x;
					\YS=2.5;
				}
				
				\begin{scope}[yshift=-\YS cm]
					
					\draw[]
					(0,0) -- (\x,0)
					;
					
					\fill
					(-1,0) circle (\rr)
					(0,0) circle (\rr)
					(1,0) circle (\rr)
					;
					
					\node at (-2.25*\x,0) {\scriptsize $Tr_2(\alpha'):$};
					
					\node at (-1*\x,-0.3*\x) {\scriptsize $1$}; 
					\node at (0*\x,-0.3*\x) {\scriptsize $2$}; 
					\node at (1*\x,-0.3*\x) {\scriptsize $3$}; 
					
				\end{scope}
				
				\begin{scope}[yshift=-1.5*\YS cm]
					
					\arcThroughThreePoints[]{1,0}{0,0.5}{-1,0};
					
					\fill
					(-1,0) circle (\rr)
					(0,0) circle (\rr)
					(1,0) circle (\rr)
					;
					
					\node at (-2.25*\x,0) {\scriptsize $Tr_3(\alpha'):$};
					
					\node at (-1*\x,-0.3*\x) {\scriptsize $2$}; 
					\node at (0*\x,-0.3*\x) {\scriptsize $3$}; 
					\node at (1*\x,-0.3*\x) {\scriptsize $4$}; 
					
				\end{scope}
				
				\begin{scope}[yshift=-2*\YS cm]
					
					\arcThroughThreePoints[]{1,0}{0,0.5}{-1,0};
					
					\fill
					(-1,0) circle (\rr)
					(0,0) circle (\rr)
					(1,0) circle (\rr)
					;
					
					\node at (-2.25*\x,0) {\scriptsize $Tr_4(\alpha'):$};
					
					\node at (-1*\x,-0.3*\x) {\scriptsize $3$}; 
					\node at (0*\x,-0.3*\x) {\scriptsize $4$}; 
					\node at (1*\x,-0.3*\x) {\scriptsize $5$}; 
					
				\end{scope}
				
			\end{tikzpicture}
			\caption{$Tr_i(\alpha')$}
			\label{Subfig-trunc-eg1-al'}
		\end{subfigure}
		\begin{subfigure}[t]{0.24\linewidth}
			\centering
			\begin{tikzpicture}[scale=0.65]
				\tikzmath{
					\x=1;
					\rr=0.08*\x;
					\YS=2.5;
				}
				
				\begin{scope}[yshift=-\YS cm]
					
					\arcThroughThreePoints[]{1,0}{0,0.5}{-1,0};
					
					\fill
					(-1,0) circle (\rr)
					(0,0) circle (\rr)
					(1,0) circle (\rr)
					;
					
					\node at (-2.25*\x,0) {\scriptsize $Tr_2(\beta):$};
					
					\node at (-1*\x,-0.3*\x) {\scriptsize $1$}; 
					\node at (0*\x,-0.3*\x) {\scriptsize $2$}; 
					\node at (1*\x,-0.3*\x) {\scriptsize $3$}; 
					
				\end{scope}
				
				\begin{scope}[yshift=-1.5*\YS cm]
					
					\arcThroughThreePoints[]{1,0}{0,0.5}{-1,0};
					
					\fill
					(-1,0) circle (\rr)
					(0,0) circle (\rr)
					(1,0) circle (\rr)
					;
					
					\node at (-2.25*\x,0) {\scriptsize $Tr_3(\beta):$};
					
					\node at (-1*\x,-0.3*\x) {\scriptsize $2$}; 
					\node at (0*\x,-0.3*\x) {\scriptsize $3$}; 
					\node at (1*\x,-0.3*\x) {\scriptsize $4$}; 
					
				\end{scope}
				
				\begin{scope}[yshift=-2*\YS cm]
					
					\arcThroughThreePoints[]{1,0}{0,0.5}{-1,0};
					
					\fill
					(-1,0) circle (\rr)
					(0,0) circle (\rr)
					(1,0) circle (\rr)
					;
					
					\node at (-2.25*\x,0) {\scriptsize $Tr_4(\beta):$};
					
					\node at (-1*\x,-0.3*\x) {\scriptsize $3$}; 
					\node at (0*\x,-0.3*\x) {\scriptsize $4$}; 
					\node at (1*\x,-0.3*\x) {\scriptsize $5$}; 
					
				\end{scope}
				
			\end{tikzpicture}
			\caption{$Tr_i(\beta)$}
			\label{Subfig-trunc-eg1-be}
		\end{subfigure}
		\begin{subfigure}[t]{0.24\linewidth}
			\centering
			\begin{tikzpicture}[scale=0.65]
				\tikzmath{
					\x=1;
					\rr=0.08*\x;
					\YS=2.5;
				}
				
				\begin{scope}[yshift=-\YS cm]
					
					\draw[]
					(0,0) -- (\x,0)
					;
					
					\fill
					(-1,0) circle (\rr)
					(0,0) circle (\rr)
					(1,0) circle (\rr)
					;
					
					\node at (-2.25*\x,0) {\scriptsize $Tr_2(\beta'):$};
					
					\node at (-1*\x,-0.3*\x) {\scriptsize $1$}; 
					\node at (0*\x,-0.3*\x) {\scriptsize $2$}; 
					\node at (1*\x,-0.3*\x) {\scriptsize $3$}; 
					
				\end{scope}
				
				\begin{scope}[yshift=-1.5*\YS cm]
					
					\arcThroughThreePoints[]{-1,0}{0,-0.5}{1,0};
					
					\fill
					(-1,0) circle (\rr)
					(0,0) circle (\rr)
					(1,0) circle (\rr)
					;
					
					\node at (-2.25*\x,0) {\scriptsize $Tr_3(\beta'):$};
					
					\node at (-1*\x,-0.3*\x) {\scriptsize $2$}; 
					\node at (0*\x,-0.3*\x) {\scriptsize $3$}; 
					\node at (1*\x,-0.3*\x) {\scriptsize $4$}; 
					
				\end{scope}
				
				\begin{scope}[yshift=-2*\YS cm]
					
					\draw[]
					(-\x,0) -- (0,0)
					;
					
					\fill
					(-1,0) circle (\rr)
					(0,0) circle (\rr)
					(1,0) circle (\rr)
					;
					
					\node at (-2.25*\x,0) {\scriptsize $Tr_4(\beta'):$};
					
					\node at (-1*\x,-0.3*\x) {\scriptsize $3$}; 
					\node at (0*\x,-0.3*\x) {\scriptsize $4$}; 
					\node at (1*\x,-0.3*\x) {\scriptsize $5$}; 
					
				\end{scope}
			\end{tikzpicture}
			\caption{$Tr_i(\beta')$}
			\label{Subfig-trunc-eg1-be'}
		\end{subfigure}
		\caption{Example of truncation}
		\label{Fig-trunc-eg1}
	\end{figure}
\end{example}

\begin{example} Figure \ref{Fig-trunc-eg1} shows two non-pointed wiggly arcs $\alpha, \alpha'$. Another two arcs $\beta,\beta'$ are added to give a dependent set.
	\begin{figure}[h!]
		\centering
		\begin{subfigure}[t]{\linewidth}
			\centering
			\begin{tikzpicture}
				\tikzmath{
					\x=1;
					\rr=0.08*\x;
					\YS=2.5;
					\XS=5;
				}
				
				\arcThroughThreePoints[dashed]{-0.025,0.27}{-1,0.75}{-2,0};
				\arcThroughThreePoints[dashed]{0.025,0.23}{1,-0.25}{2,0};
				
				\arcThroughThreePoints[dashed]{-0.025,-0.23}{-1,0.25}{-2,0};
				\arcThroughThreePoints[dashed]{0.025,-0.27}{1,-0.75}{2,0};
				
				\arcThroughThreePoints[]{0,0}{-1,0.5}{-2,0};
				\arcThroughThreePoints[]{0,0}{1,-0.5}{2,0};
				
				\fill
				(0,0) circle (\rr)
				;
				\foreach \aa in {-1,1} {
					\tikzset{xscale=\aa}
					\fill
					(\x,0) circle (\rr)
					(2*\x,0) circle (\rr)
					;
				}
				
				\node at (-1*\x, 0.6*\x) {\small $\alpha$};
				\node at (1*\x, -0.6*\x) {\small $\alpha'$};
				
				\node at (0, 0.6*\x) {\small $\beta$};
				\node at (0, -0.75*\x) {\small $\beta'$};
				
				\node at (-2*\x,-0.25*\x) {\scriptsize $1$}; 
				\node at (-1*\x,-0.25*\x) {\scriptsize $2$}; 
				\node at (0*\x,-0.25*\x) {\scriptsize $3$}; 
				\node at (1*\x,-0.25*\x) {\scriptsize $4$}; 
				\node at (2*\x,-0.25*\x) {\scriptsize $5$}; 
				
				\begin{scope}[xshift=\XS cm]
					
					\arcThroughThreePoints[red!80!black]{1,0}{0,0.5}{-1,0};
					\arcThroughThreePoints[red!80!black]{-1,0}{0,-0.5}{1,0};
					
					\draw[red!80!black]
					(-\x,0) -- (\x,0)
					;
					
					\fill[red!80!black]
					(0,0) circle (\rr)
					;
					\foreach \aa in {-1,1} {
						\tikzset{xscale=\aa}
						\fill[red!80!black]
						(\x,0) circle (\rr)
						;
					}
					
					\node at (-1*\x,-0.25*\x) {\scriptsize {\color{red!80!black}$2$}}; 
					\node at (0*\x,-0.25*\x) {\scriptsize {\color{red!80!black}$3$}}; 
					\node at (1*\x,-0.25*\x) {\scriptsize {\color{red!80!black}$4$}}; 
					
				\end{scope}
				
			\end{tikzpicture}
			\caption{}
			\label{Subfig-trunc-eg2}
		\end{subfigure}
		\begin{subfigure}[t]{0.24\linewidth}
			\centering
			\begin{tikzpicture}[scale=0.65]
				\tikzmath{
					\x=1;
					\rr=0.08*\x;
					\YS=2.5;
				}
				
				\begin{scope}[yshift=-\YS cm]
					
					\arcThroughThreePoints[]{1,0}{0,0.5}{-1,0};
					
					\fill
					(-1,0) circle (\rr)
					(0,0) circle (\rr)
					(1,0) circle (\rr)
					;
					
					\node at (-2.25*\x,0) {\scriptsize $Tr_2(\alpha):$};
					
					\node at (-1*\x,-0.3*\x) {\scriptsize $1$}; 
					\node at (0*\x,-0.3*\x) {\scriptsize $2$}; 
					\node at (1*\x,-0.3*\x) {\scriptsize $3$}; 
					
				\end{scope}
				
				\begin{scope}[yshift=-1.5*\YS cm]
					
					\draw[]
					(-\x,0) -- (0,0)
					;
					
					\fill
					(-1,0) circle (\rr)
					(0,0) circle (\rr)
					(1,0) circle (\rr)
					;
					
					\node at (-2.25*\x,0) {\scriptsize $Tr_3(\alpha):$};
					
					\node at (-1*\x,-0.3*\x) {\scriptsize $2$}; 
					\node at (0*\x,-0.3*\x) {\scriptsize $3$}; 
					\node at (1*\x,-0.3*\x) {\scriptsize $4$}; 
					
				\end{scope}
				
				\begin{scope}[yshift=-2*\YS cm]
					
					\fill
					(-1,0) circle (\rr)
					(0,0) circle (\rr)
					(1,0) circle (\rr)
					;
					
					\node at (-2.25*\x,0) {\scriptsize $Tr_4(\alpha):$};
					
					\node at (-1*\x,-0.3*\x) {\scriptsize $3$}; 
					\node at (0*\x,-0.3*\x) {\scriptsize $4$}; 
					\node at (1*\x,-0.3*\x) {\scriptsize $5$}; 
					
				\end{scope}
				
			\end{tikzpicture}
			\caption{$Tr_i(\alpha)$}
		\end{subfigure}
		\begin{subfigure}[t]{0.24\linewidth}
			\centering
			\begin{tikzpicture}[scale=0.65]
				\tikzmath{
					\x=1;
					\rr=0.08*\x;
					\YS=2.5;
				}
				
				\begin{scope}[yshift=-\YS cm]
					
					
					\fill
					(-1,0) circle (\rr)
					(0,0) circle (\rr)
					(1,0) circle (\rr)
					;
					
					\node at (-2.25*\x,0) {\scriptsize $Tr_2(\alpha'):$};
					
					\node at (-1*\x,-0.3*\x) {\scriptsize $1$}; 
					\node at (0*\x,-0.3*\x) {\scriptsize $2$}; 
					\node at (1*\x,-0.3*\x) {\scriptsize $3$}; 
					
				\end{scope}
				
				\begin{scope}[yshift=-1.5*\YS cm]
					
					\draw[]
					(0,0) -- (\x,0)
					;
					
					\fill
					(-1,0) circle (\rr)
					(0,0) circle (\rr)
					(1,0) circle (\rr)
					;
					
					\node at (-2.25*\x,0) {\scriptsize $Tr_3(\alpha'):$};
					
					\node at (-1*\x,-0.3*\x) {\scriptsize $2$}; 
					\node at (0*\x,-0.3*\x) {\scriptsize $3$}; 
					\node at (1*\x,-0.3*\x) {\scriptsize $4$}; 
					
				\end{scope}
				
				\begin{scope}[yshift=-2*\YS cm]
					
					\arcThroughThreePoints[]{-1,0}{0,-0.5}{1,0};
					
					\fill
					(-1,0) circle (\rr)
					(0,0) circle (\rr)
					(1,0) circle (\rr)
					;
					
					\node at (-2.25*\x,0) {\scriptsize $Tr_4(\alpha'):$};
					
					\node at (-1*\x,-0.3*\x) {\scriptsize $3$}; 
					\node at (0*\x,-0.3*\x) {\scriptsize $4$}; 
					\node at (1*\x,-0.3*\x) {\scriptsize $5$}; 
					
				\end{scope}
				
			\end{tikzpicture}
			\caption{$Tr_i(\alpha')$}
		\end{subfigure}
		\begin{subfigure}[t]{0.24\linewidth}
			\centering
			\begin{tikzpicture}[scale=0.65]
				\tikzmath{
					\x=1;
					\rr=0.08*\x;
					\YS=2.5;
				}
				
				\begin{scope}[yshift=-\YS cm]
					
					\arcThroughThreePoints[]{1,0}{0,0.5}{-1,0};
					
					\fill
					(-1,0) circle (\rr)
					(0,0) circle (\rr)
					(1,0) circle (\rr)
					;
					
					\node at (-2.25*\x,0) {\scriptsize $Tr_2(\beta):$};
					
					\node at (-1*\x,-0.3*\x) {\scriptsize $1$}; 
					\node at (0*\x,-0.3*\x) {\scriptsize $2$}; 
					\node at (1*\x,-0.3*\x) {\scriptsize $3$}; 
					
				\end{scope}
				
				\begin{scope}[yshift=-1.5*\YS cm]
					
					\arcThroughThreePoints[]{1,0}{0,0.5}{-1,0};
					
					\fill
					(-1,0) circle (\rr)
					(0,0) circle (\rr)
					(1,0) circle (\rr)
					;
					
					\node at (-2.25*\x,0) {\scriptsize $Tr_3(\beta):$};
					
					\node at (-1*\x,-0.3*\x) {\scriptsize $2$}; 
					\node at (0*\x,-0.3*\x) {\scriptsize $3$}; 
					\node at (1*\x,-0.3*\x) {\scriptsize $4$}; 
					
				\end{scope}
				
				\begin{scope}[yshift=-2*\YS cm]
					
					\arcThroughThreePoints[]{-1,0}{0,-0.5}{1,0};
					
					\fill
					(-1,0) circle (\rr)
					(0,0) circle (\rr)
					(1,0) circle (\rr)
					;
					
					\node at (-2.25*\x,0) {\scriptsize $Tr_4(\beta):$};
					
					\node at (-1*\x,-0.3*\x) {\scriptsize $3$}; 
					\node at (0*\x,-0.3*\x) {\scriptsize $4$}; 
					\node at (1*\x,-0.3*\x) {\scriptsize $5$}; 
					
				\end{scope}
				
			\end{tikzpicture}
			\caption{$Tr_i(\beta)$}
		\end{subfigure}
		\begin{subfigure}[t]{0.24\linewidth}
			\centering
			\begin{tikzpicture}[scale=0.65]
				\tikzmath{
					\x=1;
					\rr=0.08*\x;
					\YS=2.5;
				}
				
				\begin{scope}[yshift=-\YS cm]
					
					\arcThroughThreePoints[]{1,0}{0,0.5}{-1,0};
					
					\fill
					(-1,0) circle (\rr)
					(0,0) circle (\rr)
					(1,0) circle (\rr)
					;
					
					\node at (-2.25*\x,0) {\scriptsize $Tr_2(\beta'):$};
					
					\node at (-1*\x,-0.3*\x) {\scriptsize $1$}; 
					\node at (0*\x,-0.3*\x) {\scriptsize $2$}; 
					\node at (1*\x,-0.3*\x) {\scriptsize $3$}; 
					
				\end{scope}
				
				\begin{scope}[yshift=-1.5*\YS cm]
					
					\arcThroughThreePoints[]{-1,0}{0,-0.5}{1,0};
					
					\fill
					(-1,0) circle (\rr)
					(0,0) circle (\rr)
					(1,0) circle (\rr)
					;
					
					\node at (-2.25*\x,0) {\scriptsize $Tr_3(\beta'):$};
					
					\node at (-1*\x,-0.3*\x) {\scriptsize $2$}; 
					\node at (0*\x,-0.3*\x) {\scriptsize $3$}; 
					\node at (1*\x,-0.3*\x) {\scriptsize $4$}; 
					
				\end{scope}
				
				\begin{scope}[yshift=-2*\YS cm]
					
					\arcThroughThreePoints[]{-1,0}{0,-0.5}{1,0};
					
					\fill[]
					(-1,0) circle (\rr)
					(0,0) circle (\rr)
					(1,0) circle (\rr)
					;
					
					\node at (-2.25*\x,0) {\scriptsize $Tr_4(\beta'):$};
					
					\node at (-1*\x,-0.3*\x) {\scriptsize $3$}; 
					\node at (0*\x,-0.3*\x) {\scriptsize $4$}; 
					\node at (1*\x,-0.3*\x) {\scriptsize $5$}; 
					
				\end{scope}
				
			\end{tikzpicture}
			\caption{$Tr_i(\beta')$}
		\end{subfigure}
		\caption{Example of truncation}
	\end{figure}
	
	For this example, 
	\begin{align*}
		Tr_i(\alpha) + Tr_i(\alpha')  = \tfrac{1}{2}( Tr_i(\beta) + Tr_i(\beta') ),
	\end{align*}
	for $i=2,4$. For $i=3$, we actually have ``no cancellation'' in the coloured picture in Figure \ref{Subfig-trunc-eg2}.
\end{example}

We want to establish the following statement.

\begin{lemma}
	For two incompatible arcs $\alpha, \alpha'$, let $\beta, \beta'$ to be the choices for the upper and lower hulls to form a circuit. 
	\begin{itemize}
		\item $\alpha, \alpha'$ are crossing arcs: define the truncation decomposition $D(\alpha)$ (likewise for $D(\alpha')$) to be the formal sum
		\[
		D(\alpha) := \sum_{i=2}^{n-1} Tr_i(\alpha), 
		\]
		then for $i \in [2, n-1]$, we have 
		\[
		Tr_i (\alpha) + Tr_i (\alpha') =  Tr_i (\beta) + Tr_i (\beta').
		\]
		\item $\alpha, \alpha'$ are non-pointed arcs: suppose $j+1$ is their common endpoint, then for $i \neq j$
		\begin{align*}
			2(Tr_i (\alpha) + Tr_i (\alpha')) = Tr_i (\beta) + Tr_i (\beta');
		\end{align*}
		and for $i = j$,
		\begin{align*}
			\begin{cases}
				Tr_i (\alpha) + Tr_i (\alpha') = l + r,  \\  
				Tr_i (\beta) + Tr_i (\beta') = u + d.
			\end{cases}\quad \text{for } i = j;
		\end{align*}
	\end{itemize}
\end{lemma}

The following lemma ensures that the hypotheses need only to be satisfied for the non-face $\{(l,i),(r,i)\}$:
%
\begin{lemma}
\label{l:truncation-works}
This decomposition $D$ has the following property. For every non-face $F=\{\alpha, \alpha'\}$, let $\beta$ (resp. $\beta'$) denote the upper hull (resp. lower hull) of $\alpha$ and $\alpha'$. Then:
\begin{itemize}
\item If $\alpha$ and $\alpha'$ are crossing, then $\pth{D(\alpha)+D(\alpha')}-\pth{ D(\beta)+D(\beta')}=0$,
\item Otherwise, $\alpha$ and $\alpha'$ are pointed at node $i$, and
\[
2\pth{D(\alpha)+D(\alpha')}-\pth{ D(\beta)+D(\beta')}=2\pth{(l,i)+(r,i)} - \pth{(u,i)+(d,i)}.
\]
\end{itemize} 
\end{lemma}
\medskip

Define $\widetilde R$ on $\widetilde {\wigglyArcs_P}$, so that the only linear dependences among its values are:
\[
2\pth{\widetilde R((l,i))+\widetilde R((r,i))} - \pth{\widetilde R((u,i))+\widetilde R((d,i))}=0, \qquad \forall i\in\{2,\ldots,n-1\}.
\]
For example, define $\widetilde R$ on $\widetilde {\wigglyArcs_P}$ as described in Definition~\ref{d:rigidity-vector}. 
%
Extend $\widetilde R$ linearly on $\Z[\widetilde V]$, and define
\[
R:\wigglyArcs_P\to \R^d, \alpha\mapsto \widetilde R(D(\alpha)).
\]
By Lemma~\ref{l:truncation-works} and linearity of $\widetilde R$, $R$ satisfies the hypotheses of Theorem~\ref{thm:simplicial-complex-is-regular-triangulation}. Indeed, for every non-face $F=\{\alpha, \alpha'\}$ (with $\beta,\beta'$ the lower and upper hull of $\alpha,\alpha'$):
\begin{itemize}
\item If $\alpha$ and $\alpha'$ are crossing, then 
\begin{align*}
\pth{R(\alpha)+R(\alpha')}-\pth{ R(\beta)+R(\beta')} &= \pth{\widetilde R(D(\alpha))+\widetilde R(D(\alpha')}-\pth{ \widetilde R(D(\beta)+\widetilde R(D(\beta')}\\
&= \widetilde R \pth{\pth{D(\alpha)+D(\alpha')}-\pth{ D(\beta)+D(\beta')}} \\
&=0,
\end{align*}
\item Otherwise, $\alpha$ and $\alpha'$ are pointed at node $i$, and
\begin{align*}
2\pth{R(\alpha)+R(\alpha')}-\pth{ R(\beta)+R(\beta')}
&= 2\pth{\widetilde R(D(\alpha))+\widetilde R(D(\alpha'))}-\pth{ \widetilde R(D(\beta))+\widetilde R(D(\beta'))}\\
&=\widetilde R \pth{2\pth{D(\alpha)+D(\alpha')}-\pth{ D(\beta)+D(\beta')}} \\
&= \widetilde R \pth{2\pth{(l,i)+(r,i)} - \pth{(u,i)+(d,i)}}\\
&=0.
\end{align*}
\end{itemize}
In both cases, $\{\beta,\beta'\}$ is a compatible pair of s and therefore a face of $WC_P$. It remains to specify a height function $f:\wigglyArcs_P\to \R$ before fully satisfying the hypotheses of Theorem~\ref{thm:simplicial-complex-is-regular-triangulation}.


 %we introduce the
%%%%%%%%%%%%%%%%%%%%%%%%%%%%%%%%%%%%%%

\bibliographystyle{alpha}
\bibliography{biblio}

%%%%%%%%%%%%%%%%%%%%%%%%%%%%%%%%%%%%%%

\end{document}
